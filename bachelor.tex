\documentclass[bachelor,winfonts]{jnuthesis}

% 论文标题
\titlea{人工智能应用——聊天机器人的开发}
\titleb{}

% 论文作者姓名
\author{谭啸}
% 论文作者学生证号
\studentnum{1030413620}
% 导师姓名职称
\supervisor{徐华}
\supervisorpos{教授}
% 第二行导师姓名职称,仿照第一行填写,没有则留空
\supervisorb{}
\supervisorbpos{}
% 论文作者的学科与专业方向
\major{计算机科学与技术}
% 论文作者所在院系的中文名称,学士学位论文此处不带“学院”二字
\department{物联网}
% 论文作者所在学校或机构的名称。此属性可选,默认值为``江南大学''。
\institute{江南大学}
% 学士学位获得日期,需设置年、月,默认为编译日期。
%\bachelordegreeyear{2017}
%\bachelordegreemonth{6}

\begin{document}

% 制作中文封面
\maketitle

% 开始前言部分
\frontmatter

% 论文的中文摘要
\begin{abstract}

\keywords{小世界理论;网络模型;数据中心}
\end{abstract}

% 论文的英文摘要
\begin{englishabstract}

% 英文关键词。关键词之间用英文半角逗号隔开,末尾无符号。
\englishkeywords{Small World, Network Model, Data Center}
\end{englishabstract}

% 生成论文目录
\tableofcontents

% 开始正文部分
\mainmatter

\chapter{绪论}\label{chapter_introduction}

\section{研究背景}
\section{国内外研究现状}
\section{存在的主要问题}
\section{论文主要研究内容}


\chapter{相关理论基础}


\chapter{AIML核心机制}


\chapter{中文分词}


\chapter{系统设计与实现}


\chapter{总结与展望}
\section{论文总结}
\section{工作展望}


\begin{acknowledgement}
  首先感谢
\end{acknowledgement}


% 参考文献
\nocite{*}
\bibliography{bachelor}
%%%%%%%%%%%%%%%%%%%%%%%%%%%%%%%%%%%%%%%%%%%%%%%%%%%%%%%%%%%%%%%%%%%%%%%%%%%%%%%
\end{document}