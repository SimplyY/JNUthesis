%% 使用 jnuthesis 文档类生成南京大学学位论文的示例文档
%%
%% 作者:胡海星,starfish (at) gmail (dot) com
%% 项目主页: http://haixing-hu.github.io/jnu-thesis/
%%
%% 本样例文档中用到了吕琦同学的博士论文的提高和部分内容,在此对他表示感谢。
%%
\documentclass[bachelor,winfonts]{jnuthesis}
%% jnuthesis 文档类的可选参数有:
%%   nobackinfo 取消封二页导师签名信息。注意,按照南大的规定,是需要签名页的。
%%   phd/master/bachelor 选择博士/硕士/学士论文

% 使用 blindtext 宏包自动生成章节文字
% 这仅仅是用于生成样例文档,正式论文中一般用不到该宏包
\usepackage[math]{blindtext}

%%%%%%%%%%%%%%%%%%%%%%%%%%%%%%%%%%%%%%%%%%%%%%%%%%%%%%%%%%%%%%%%%%%%%%%%%%%%%%%
% 设置论文的中文封面

% 如果论文标题过长,可以分两行,第一行用\titlea{}定义,第二行用\titleb{}定义,将上面的\title{}注释掉
\titlea{半轻衰变$D^+\to \omega(\phi)e^+\nu_e$的研究}
\titleb{和弱衰变$J/\psi \to D_s^{(*)-}e^+\nu_e$的寻找}

% 论文作者姓名
\author{韦小宝}
% 论文作者学生证号
\studentnum{1234567890}
% 导师姓名职称
\supervisor{陈近南}
\supervisorpos{教授}
% 第二行导师姓名职称,仿照第一行填写,没有则留空
\supervisorb{}
\supervisorbpos{}
% 论文作者的学科与专业方向
\major{数字媒体技术}
% 论文作者所在院系的中文名称,学士学位论文此处不带“学院”二字
\department{数字媒体}
% 论文作者所在学校或机构的名称。此属性可选,默认值为``江南大学''。
\institute{江南大学}
% 学士学位获得日期,需设置年、月,默认为编译日期。
%\bachelordegreeyear{2017}
%\bachelordegreemonth{6}

%%%%%%%%%%%%%%%%%%%%%%%%%%%%%%%%%%%%%%%%%%%%%%%%%%%%%%%%%%%%%%%%%%%%%%%%%%%%%%%
\begin{document}

%%%%%%%%%%%%%%%%%%%%%%%%%%%%%%%%%%%%%%%%%%%%%%%%%%%%%%%%%%%%%%%%%%%%%%%%%%%%%%%

% 制作中文封面
\maketitle

%%%%%%%%%%%%%%%%%%%%%%%%%%%%%%%%%%%%%%%%%%%%%%%%%%%%%%%%%%%%%%%%%%%%%%%%%%%%%%%
% 开始前言部分
\frontmatter

%%%%%%%%%%%%%%%%%%%%%%%%%%%%%%%%%%%%%%%%%%%%%%%%%%%%%%%%%%%%%%%%%%%%%%%%%%%%%%%
% 论文的中文摘要
\begin{abstract}
复杂网络的研究可上溯到20世纪60年代对ER网络的研究。90年后代随着Internet
的发展,以及对人类社会、通信网络、生物网络、社交网络等各领域研究的深入,
发现了小世界网络和无尺度现象等普适现象与方法。对复杂网络的定性定量的科
学理解和分析,已成为如今网络时代科学研究的一个重点课题。

在此背景下,由于云计算时代的到来,本文针对面向云计算的数据中心网络基础
设施设计中的若干问题,进行了几方面的研究。………………
% 中文关键词。关键词之间用中文全角分号隔开,末尾无标点符号。
\keywords{小世界理论;网络模型;数据中心}
\end{abstract}

%%%%%%%%%%%%%%%%%%%%%%%%%%%%%%%%%%%%%%%%%%%%%%%%%%%%%%%%%%%%%%%%%%%%%%%%%%%%%%%
% 论文的英文摘要
\begin{englishabstract}
\blindtext
% 英文关键词。关键词之间用英文半角逗号隔开,末尾无符号。
\englishkeywords{Small World, Network Model, Data Center}
\end{englishabstract}

%%%%%%%%%%%%%%%%%%%%%%%%%%%%%%%%%%%%%%%%%%%%%%%%%%%%%%%%%%%%%%%%%%%%%%%%%%%%%%%
% 生成论文目次
\tableofcontents

%%%%%%%%%%%%%%%%%%%%%%%%%%%%%%%%%%%%%%%%%%%%%%%%%%%%%%%%%%%%%%%%%%%%%%%%%%%%%%%
% 开始正文部分
\mainmatter

%%%%%%%%%%%%%%%%%%%%%%%%%%%%%%%%%%%%%%%%%%%%%%%%%%%%%%%%%%%%%%%%%%%%%%%%%%%%%%%
% 学位论文的正文应以《绪论》作为第一章
\chapter{绪论}\label{chapter_introduction}
\section{研究背景}

在分布式网络领域,沿着高性能集群、普世计算、网格计算的方向,现已走入云
计算时代。

云计算对信息技术架构造成了越来越大的影响。例如,借助Amazon EC2云平台,
用户借助其基础设施,可以十分方便的部署各类应用,以支持企业服务需求。用
户可以按需购买计算资源,网络带宽,存储空间等各类资源以支持他们的业务需
求,并在业务完成之后迅速的归还这些资源。通过云技术,用户可以集中在他们
擅长的核心业务之中,而不会被诸如硬件购买、安装系统、网络设置、备份和安
全等等问题干扰。

与此同时,随着计算机的普及化和微型化,现在的手持设备拥有不输于7年前台式
机的处理能力。在网络时代面前,智能终端广泛普及,每个人都可成为信息源。
在信息爆炸的时代,数据挖掘、机器学习、金融分析和模拟等行业中不断涌现新
的需求,诸如针对用户行为和社会关系的挖掘进行广告精准投放,用户行为预测
等。为了支撑PB级尺度的数据规模,需要海量的计算节点,催生并不断促进了各
行各业对云计算基础设施的建设需求。

海量的数据需要海量的处理能力,然而海量的处理能力又需要高带宽的网络IO为
承载。作为云环境中最基础的一环,IaaS层在网络、存储、计算资源的分割这几
方面,承担起整个系统的基石。虽然并非必须,但一般来说,为考虑沙盘环境,
以及对资源的细粒度切割分配,IaaS通常会伴随着虚拟化技术的运用。虚拟化具
有许多与云计算切合的特点,例如,虚拟化可以屏蔽物理环境的差异,可在多物
理节点中进行无缝迁移,可对系统进行快照和还原。这些特点都与云计算时代所
追求的灵活性、高伸缩性、快速响应等特点而吻合。

在虚拟化实现方面,目前已取得了诸如ESXi,Xen,KVM等成熟成果。然而当虚拟
化扩大的一定规模,随着节点数目的增多,在网络方面将会面临一系列取舍的问
题。例如,基于二层交换的扁平网络,当节点数目上升到千数量级时,广播报文
将会极大的拖累网络性能,必须通过划分子网,通过三层路由等形式重新规划为
多层网络结构;另一方面,除了联通之外,还需要考虑ACL控制,负载均衡,外网
通讯等各类防火墙以及NAT规则的实现。这些复杂的网络配置,在一定程度上抵消
了虚拟机带来的灵活性。例如虽然虚拟机可根据需要动态迁移,但在迁移之后,
由于网络位置的变化需要重新进行网络参数配置。虽然虚拟机在迁移过程中系统
内部状态没有变化,但站在网络角度看,该虚拟节点跟关机重启没有区别。

针对上述问题,本文站在面向云计算时代的数据中心网络建设的角度,对网络模
型进行深入研究和探讨。通过改善二层交换网络的ARP机制来解决广播风暴问题,
引入比树形网络更为复杂的复杂网络理论,指导网络节点的互联模型。从而将网
络的复杂性隐藏在节点环境之外,在节点层面仅提供简单但巨大的二层交换扁平
网络。

\section{研究目的与意义}
\subsection{现有解决方法}
\Blindtext
\begin{table}
  \centering
  \begin{tabular}{cccp{38mm}}
    \toprule
    \textbf{文档域类型} & \textbf{Java类型} & \textbf{宽度(字节)} & \textbf{说明} \\
    \midrule
    BOOLEAN  & boolean &  1  & \\
    CHAR     & char    &  2  & UTF-16字符 \\
    BYTE     & byte    &  1  & 有符号8位整数 \\
    SHORT    & short   &  2  & 有符号16位整数 \\
    INT      & int     &  4  & 有符号32位整数 \\
    LONG     & long    &  8  & 有符号64位整数 \\
    STRING   & String  &  字符串长度  & 以UTF-8编码存储 \\
    DATE     & java.util.Date & 8 & 距离GMT时间1970年1月1日0点0分0秒的毫秒数 \\
    BYTE\_ARRAY & byte$[]$ & 数组长度 & 用于存储二进制值 \\
    BIG\_INTEGER & java.math.BigInteger & 和具体值有关 & 任意精度的长整数 \\
    BIG\_DECIMAL & java.math.BigDecimal & 和具体值有关 & 任意精度的十进制实数 \\
    \bottomrule
  \end{tabular}
  \caption{测试表格}\label{table:test1}
\end{table}
\Blindtext
\subsection{现有问题与不足}

测试一下脚注\footnote{测试脚注},测试一下脚注\footnote{测试脚注},测试一下脚
注\footnote{测试脚注},测试一下脚注\footnote{测试脚注},测试一下脚注\footnote{测
  试脚注},测试一下脚注\footnote{测试脚注},测试一下脚注\footnote{测试脚注},测
试一下脚注\footnote{测试脚注},测试一下脚注\footnote{测试脚注},测试一下脚
注\footnote{测试脚注}。

测试一下引用\cite{newman2006structure},连续引用
\cite{newman2001random,aiello2000random,bollobas2001random},另一个连续引用
\cite{newman2001random,bollobas2001random,barabasi1999emergence}。测试一下带页码
的引用\cite[124--128]{erdHos1961strength}。

下面是一个项目列表:

\begin{itemize}
\item 这是第一项。这是第一项。
\item 这是第二项。这是第二项。这是第二项。这是第二项。这是第二项。这是第二项。这
  是第二项。这是第二项。这是第二项。这是第二项。这是第二项。
\item 这是第三项。这是第三项。这是第三项。
  \begin{itemize}
  \item 测试第二层列表。测试第二层列表。
  \item 测试第二层列表。测试第二层列表。
  \begin{itemize}
     \item 测试第三层列表。测试第三层列表。
     \item 测试第三层列表。测试第三层列表。
  \end{itemize}
  \item 测试第二层列表。测试第二层列表。测试第二层列表。测试第二层列表。测试第二
    层列表。
  \end{itemize}
\item 这是第四项。这是第四项。这是第四项。
  \begin{enumerate}
  \item 测试第二层列表。测试第二层列表。测试第二层列表。测试第二层列表。测试第二
    层列表。测试第二层列表。测试第二层列表。测试第二层列表。
  \item 测试第二层列表。测试第二层列表。
  \item 测试第二层列表。测试第二层列表。测试第二层列表。测试第二层列表。测试第二
    层列表。
  \end{enumerate}
\end{itemize}

下面是一个编号列表:

\begin{enumerate}
\item 这是第一项。这是第一项。这是第一项。这是第一项。这是第一项。这是第一项。这
  是第一项。这是第一项。这是第一项。这是第一项。这是第一项。
\item 这是第二项。这是第二项。
\item 这是第三项。这是第三项。这是第三项。
  \begin{itemize}
  \item 测试第二层列表。测试第二层列表。
  \item 测试第二层列表。测试第二层列表。
  \item 测试第二层列表。测试第二层列表。测试第二层列表。测试第二层列表。测试第二
    层列表。
  \end{itemize}
\item 这是第四项。这是第四项。这是第四项。
  \begin{enumerate}
  \item 测试第二层列表。测试第二层列表。
  \begin{enumerate}
  \item 测试第三层列表。测试第三层列表。测试第三层列表。测试第三层列表。测试第三
    层列表。测试第三层列表。
  \item 测试第三层列表。测试第三层列表。
  \item 测试第三层列表。测试第三层列表。测试第三层列表。
  \end{enumerate}
  \item 测试第二层列表。测试第二层列表。测试第二层列表。
  \end{enumerate}
\end{enumerate}

下面是最多三层的阿拉伯数字列表:
\begin{arabicenum}
\item 第1项
\item 第2项
  \begin{arabicenum}
  \item 第2.1项
  \item 第2.2项
    \begin{arabicenum}
    \item 第2.2.1项
    \item 第2.2.2项
    \item 第2.2.3项
    \end{arabicenum}
  \item 第2.3项
  \end{arabicenum}
\item 第3项
\end{arabicenum}

下面是最多两层的罗马数字列表:
\begin{romanenum}
\item 第1项
\item 第2项
  \begin{romanenum}
  \item 第2.1项
  \item 第2.2项
  \item 第2.3项
  \end{romanenum}
\item 第3项
\end{romanenum}

下面是最多两层的小写字母列表:
\begin{alphaenum}
\item 第1项
\item 第2项
  \begin{alphaenum}
  \item 第2.1项
  \item 第2.2项
  \item 第2.3项
  \end{alphaenum}
\item 第3项
\end{alphaenum}

下面是最多两层的情况列表:
\begin{caseenum}
\item 第1项
\item 第2项
  \begin{caseenum}
  \item 第2.1项
  \item 第2.2项
  \item 第2.3项
  \end{caseenum}
\item 第3项
\end{caseenum}

下面是最多两层的步骤列表:
\begin{stepenum}
\item 第1项
\item 第2项
  \begin{stepenum}
  \item 第2.1项
  \item 第2.2项
  \item 第2.3项
  \end{stepenum}
\item 第3项
\end{stepenum}

下面测试一下引用环境|quote|。下面测试一下引用环境|quote|。下面测试一下引用环境|quote|。
下面测试一下引用环境|quote|。下面测试一下引用环境|quote|。下面测试一下引用环境|quote|。
下面测试一下引用环境|quote|。下面测试一下引用环境|quote|。下面测试一下引用环境|quote|。

\begin{quote}
这是一段引用。这是一段引用。这是一段引用。这是一段引用。这是一段引用。这是一段引用。
这是一段引用。这是一段引用。这是一段引用。这是一段引用。这是一段引用。这是一段引用。
这是一段引用。这是一段引用。这是一段引用。这是一段引用。这是一段引用。

这是一段引用。这是一段引用。这是一段引用。这是一段引用。这是一段引用。这是一段引用。
这是一段引用。这是一段引用。这是一段引用。

这是一段引用。这是一段引用。
\end{quote}

下面测试一下引用环境|quotation|。下面测试一下引用环境|quotation|。
下面测试一下引用环境|quotation|。下面测试一下引用环境|quotation|。
下面测试一下引用环境|quotation|。下面测试一下引用环境|quotation|。
下面测试一下引用环境|quotation|。下面测试一下引用环境|quotation|。
下面测试一下引用环境|quotation|。

\begin{quotation}
这是一段引用。这是一段引用。这是一段引用。这是一段引用。这是一段引用。这是一段引用。
这是一段引用。这是一段引用。这是一段引用。这是一段引用。这是一段引用。这是一段引用。
这是一段引用。这是一段引用。这是一段引用。这是一段引用。这是一段引用。

这是一段引用。这是一段引用。这是一段引用。这是一段引用。这是一段引用。这是一段引用。
这是一段引用。这是一段引用。这是一段引用。

这是一段引用。这是一段引用。
\end{quotation}

引用结束。引用结束。引用结束。引用结束。引用结束。引用结束。引用结束。引用结束。引用结束。
引用结束。引用结束。引用结束。

测试一下定理环境。

\begin{theorem}[测试定理]
测试一下定理环境。测试一下定理环境。测试一下定理环境。测试一下定理环境。测试一下
定理环境。测试一下定理环境。测试一下定理环境。
\end{theorem}
\begin{proof}
\blindtext
\end{proof}

\blindtext

\begin{theorem}
测试一下定理环境。测试一下定理环境。测试一下定理环境。测试一下定理环境。测试一下
定理环境。测试一下定理环境。测试一下定理环境。
\end{theorem}
\begin{proof}
\blindtext
\end{proof}

\blindtext

\begin{lemma}
测试一下定理环境。测试一下定理环境。测试一下定理环境。测试一下定理环境。测试一下
定理环境。测试一下定理环境。测试一下定理环境。
\end{lemma}
\begin{proof}
\blindtext
\end{proof}

\blindtext

\begin{definition}
测试一下定理环境。测试一下定理环境。测试一下定理环境。测试一下定理环境。测试一下
定理环境。测试一下定理环境。测试一下定理环境。
\end{definition}

\blindtext

\begin{corollary}
测试一下定理环境。测试一下定理环境。测试一下定理环境。测试一下定理环境。测试一下
定理环境。测试一下定理环境。测试一下定理环境。
\end{corollary}

\blindtext

\begin{proposition}
测试一下定理环境。测试一下定理环境。测试一下定理环境。测试一下定理环境。测试一下
定理环境。测试一下定理环境。测试一下定理环境。
\end{proposition}

\blindtext

\begin{fact}
测试一下定理环境。测试一下定理环境。测试一下定理环境。测试一下定理环境。测试一下
定理环境。测试一下定理环境。测试一下定理环境。
\end{fact}

\blindtext

\begin{assumption}
测试一下定理环境。测试一下定理环境。测试一下定理环境。测试一下定理环境。测试一下
定理环境。测试一下定理环境。测试一下定理环境。
\end{assumption}

\blindtext

\begin{conjecture}
测试一下定理环境。测试一下定理环境。测试一下定理环境。测试一下定理环境。测试一下
定理环境。测试一下定理环境。测试一下定理环境。
\end{conjecture}

\blindtext

\begin{hypothesis}
测试一下定理环境。测试一下定理环境。测试一下定理环境。测试一下定理环境。测试一下
定理环境。测试一下定理环境。测试一下定理环境。
\end{hypothesis}

\blindtext

\begin{axiom}
测试一下定理环境。测试一下定理环境。测试一下定理环境。测试一下定理环境。测试一下
定理环境。测试一下定理环境。测试一下定理环境。
\end{axiom}

\blindtext

\begin{postulate}
测试一下定理环境。测试一下定理环境。测试一下定理环境。测试一下定理环境。测试一下
定理环境。测试一下定理环境。测试一下定理环境。
\end{postulate}

\blindtext

\begin{principle}
测试一下定理环境。测试一下定理环境。测试一下定理环境。测试一下定理环境。测试一下
定理环境。测试一下定理环境。测试一下定理环境。
\end{principle}

\blindtext

\begin{problem}
测试一下定理环境。测试一下定理环境。测试一下定理环境。测试一下定理环境。测试一下
定理环境。测试一下定理环境。测试一下定理环境。
\end{problem}
\begin{solution}
\blindtext
\end{solution}

\blindtext

\begin{problem}
测试一下定理环境。测试一下定理环境。测试一下定理环境。测试一下定理环境。测试一下
定理环境。测试一下定理环境。测试一下定理环境。
\end{problem}
\begin{solution}
\blindtext
\end{solution}

\blindtext

\begin{exercise}
测试一下定理环境。测试一下定理环境。测试一下定理环境。测试一下定理环境。测试一下
定理环境。测试一下定理环境。测试一下定理环境。
\end{exercise}

\blindtext

\begin{exercise}
测试一下定理环境。测试一下定理环境。测试一下定理环境。测试一下定理环境。测试一下
定理环境。测试一下定理环境。测试一下定理环境。
\end{exercise}

\begin{algorithm}
测试一下定理环境。测试一下定理环境。测试一下定理环境。测试一下定理环境。测试一下
定理环境。测试一下定理环境。测试一下定理环境。
\end{algorithm}

\subsection{中心观点与思想}

云计算在概念上通常被分为IaaS、PaaS、SaaS几个层面。但透过分类去理解其本
质,可认为是上世纪70年代基于大型计算机的中心控制型瘦客户端终端模式,在
如今技术水平上的一种新的表达,是在技术发展道路中,螺旋上升的结果。

与瘦客户端相比,云计算在设计结构上存在一定的相似性。

\begin{enumerate}
\item 中心控制的模式:通过中心的大规模硬件提供统一的计算,可大大降低管理成本,提
  高硬件资源利用率,同时降低客户端的硬件成本需求。例如Nvidia推出Georce GRID平台
  \cite{NVIDIAGRID},推出了GaaS\footnote{Gaming as a Service}概念。将
  GPU放置在云端,使得用户不需要再不断购买升级显卡,并可在更为广泛的终端(包括手机、
  平板、智能电视)和地点体验最新的游戏。
\item 数据集中:由于瘦客户端的关系,数据都集中存储在中心,可对数据提供
  可靠的保护,并且通过按需调用的实现方式,降低对网络带宽的需求。
\end{enumerate}

在设计思路上,两者都为了降低管理成本和硬件成本、以低能耗、高弹性等需求
为设计目标。随着技术的进步,云计算在具体实现形态上与传统的大型机也有很
大的不同:

一方面,云中心不再是传统的一台大型机,而是用大量廉价计算节点的互联来提
供海量资源。云计算更强调资源规模的无缝、平滑扩展,以及高可靠性,无单点
故障问题。另一方面,云计算时代的终端,也具备相当计算能力。随着web2.0的
整合,还有向胖客户端和智能终端发展的趋势。

总而言之,云计算在大框架中是传统的中心控制/终端的模式,但在中心与终端
两方面,都引入分布式技术加以改良。核心的思路是在低成本的前提下做到高可
靠性、高灵活性和高伸缩性。因此,云计算并不仅仅以数量换性能的表象,本质
上为低成本高性能,追求高能效比,并在实现层面讲究可实现性和可操作性。
\subsection{需要解决的问题与挑战}

测试一下中文字体:

{\songti\zihao{0} 宋体,初号}

{\songti\zihao{-0} 宋体,小初}

{\songti\zihao{1} 宋体,一号}

{\songti\zihao{-1} 宋体,小一}

{\songti\zihao{2} 宋体,二号}

{\songti\zihao{-2} 宋体,小二}

{\songti\zihao{3} 宋体,三号}

{\songti\zihao{-3} 宋体,小三}

{\songti\zihao{4} 宋体,四号}

{\songti\zihao{-4} 宋体,小四}

{\songti\zihao{5} 宋体,五号}

{\songti\zihao{-5} 宋体,小五}

{\songti\zihao{6} 宋体,六号}

{\songti\zihao{-6} 宋体,小六}

{\songti\zihao{7} 宋体,七号}

{\songti\zihao{8} 宋体,八号}

{\heiti\zihao{0} 黑体,初号}

{\heiti\zihao{-0} 黑体,小初}

{\heiti\zihao{1} 黑体,一号}

{\heiti\zihao{-1} 黑体,小一}

{\heiti\zihao{2} 黑体,二号}

{\heiti\zihao{-2} 黑体,小二}

{\heiti\zihao{3} 黑体,三号}

{\heiti\zihao{-3} 黑体,小三}

{\heiti\zihao{4} 黑体,四号}

{\heiti\zihao{-4} 黑体,小四}

{\heiti\zihao{5} 黑体,五号}

{\heiti\zihao{-5} 黑体,小五}

{\heiti\zihao{6} 黑体,六号}

{\heiti\zihao{-6} 黑体,小六}

{\heiti\zihao{7} 黑体,七号}

{\heiti\zihao{8} 黑体,八号}

{\kaishu\zihao{0} 楷书,初号}

{\kaishu\zihao{-0} 楷书,小初}

{\kaishu\zihao{1} 楷书,一号}

{\kaishu\zihao{-1} 楷书,小一}

{\kaishu\zihao{2} 楷书,二号}

{\kaishu\zihao{-2} 楷书,小二}

{\kaishu\zihao{3} 楷书,三号}

{\kaishu\zihao{-3} 楷书,小三}

{\kaishu\zihao{4} 楷书,四号}

{\kaishu\zihao{-4} 楷书,小四}

{\kaishu\zihao{5} 楷书,五号}

{\kaishu\zihao{-5} 楷书,小五}

{\kaishu\zihao{6} 楷书,六号}

{\kaishu\zihao{-6} 楷书,小六}

{\kaishu\zihao{7} 楷书,七号}

{\kaishu\zihao{8} 楷书,八号}

{\fangsong\zihao{0} 仿宋,初号}

{\fangsong\zihao{-0} 仿宋,小初}

{\fangsong\zihao{1} 仿宋,一号}

{\fangsong\zihao{-1} 仿宋,小一}

{\fangsong\zihao{2} 仿宋,二号}

{\fangsong\zihao{-2} 仿宋,小二}

{\fangsong\zihao{3} 仿宋,三号}

{\fangsong\zihao{-3} 仿宋,小三}

{\fangsong\zihao{4} 仿宋,四号}

{\fangsong\zihao{-4} 仿宋,小四}

{\fangsong\zihao{5} 仿宋,五号}

{\fangsong\zihao{-5} 仿宋,小五}

{\fangsong\zihao{6} 仿宋,六号}

{\fangsong\zihao{-6} 仿宋,小六}

{\fangsong\zihao{7} 仿宋,七号}

{\fangsong\zihao{8} 仿宋,八号}

测试一下标准字号:

{\Huge 汉字,Huge}

{\huge 汉字,huge}

{\LARGE 汉字,LARGE}

{\Large 汉字,Large}

{\large 汉字,large}

{\normalsize 汉字,normalsize}

{\small 汉字,small}

{\footnotesize 汉字,footnotesize}

{\scriptsize 汉字,scriptsize}

{\tiny 汉字,tiny}

测试一下标准字体的变形:

{\songti 宋体} {\heiti 黑体} {\kaishu 楷书} {\fangsong 仿宋}

{\textsl{textsl字体}}

{\bfseries bfseries字体}

{\textbf{textbf字体}}

{\textit{textit字体}}

测试一下数学公式中的字体大小。

\newcommand{\set}[1]{\left\{\,#1\,\right\}}
\newcommand{\card}[1]{\left|\,#1\,\right|}

Fall-Out指标计算公式如下:
\begin{equation*}
  \mbox{fallout} = \frac{\card{\set{\text{不相关文档}}\cap\set{\text{获取的文档}}}}{\card{\set{\text{不相关文档}}}}
\end{equation*}

\section{研究的应用背景}
云计算作为分布式技术的当前表现形式,通过将众多节点资源整合,以冗余、去
中心化的分布式模式,实现传统技术中需要大型机才能解决的海量信息问题。一
言而概之,“人多力量大”。

但随着节点数目的增多,问题的重点将逐步转换为如何对大量节点进行高效互联。图
\ref{fig:test1}所示为传统IaaS云中心网络结构的一部分。通过BR边界路由器,AR接入路
由器构建数据中心的主干;核心交换机和接入交换机S,构成二层交换网络层,大量的服务
器节点通过二层交换机被最终接入整个网络。
\begin{figure}[htbp]
  \centering
  \includegraphics[width= 0.5\textwidth]{jnuname.eps}\\
  \caption{测试插图}\label{fig:test1}
\end{figure}
在上述传统网络中,当节点总数达到数千乃至万数量级时,上层链路的聚合带宽
将不断提高,从而对核心交换机、接入路由器、边界路由器的指标提出了极高的
要求。以至于少数核心网络设备,成为整个网络中的高价格、高性能单点。一方
面与原本追求低成本、分布化、去中心化的云技术设计理念背道而驰,另一方面
也降低了网络的健壮性。因而本文所做工作对数据中心的网络基础设施而言,具
有较大应用价值。
\subsection{IaaS云中心}
\Blindtext
\subsection{PaaS云中心}
\Blindtext
\section{论文结构}
\Blindtext

%%%%%%%%%%%%%%%%%%%%%%%%%%%%%%%%%%%%%%%%%%%%%%%%%%%%%%%%%%%%%%%%%%%%%%%%%%%%%%%
\chapter{小世界网络模型}\label{chapter_smallworld}
\section{小世界现象}

生活中,常出现初次见面的陌生人却拥有双方都认识的共同熟人,于是大家时常
会感叹:“这世界真小!”。这种现象被称为``小世界现象'',后又称为``六度
分割理论''。

\begin{figure}[htbp]
  \centering
  \includegraphics[width= 0.5\textwidth]{jnuname.eps}\\
  \caption{测试插图}\label{fig:test2}
\end{figure}

1909年,现代无线电之父Guglielmo Marconi在其诺贝尔奖致辞中讨论了覆盖整个地球所需
的无线电中继站数目,并根据他的实验结果计算出平均需要$5.83$(近似为$6$)个中继站
\cite{marconi1909nobel}。这个结论,被认为是``六度分割理论''中的常数$6$的最早出处
\cite{barabasi2003linked}。

1929年,匈牙利作家Frigyes Karinthy发表了一部短篇小说集《Everything is
  Different》。其中一篇名为《Chain-Links》的小说以抽象的、概念性的和虚
构的方式研究了网络理论领域的很多问题,而这些问题使得未来几代的数学家、
社会学家和物理学家都为之着迷\cite{newman2006structure,
  barabasi2003linked}。Karinthy认为,随着通讯技术和交通技术的发展,人际
关系网会变得越来越来大,扩张得越来越远,整个世界将因此而``缩小''。他认
为,虽然人类个体之间的物理距离可能很远,但人类社交网络密度的增加使得人
类个体之间的社会性距离变得非常小。根据这个假设,Karinthy的小说的主角相
信:``任何两个人之间可以通过不超过五个中间人相联系''。Karinthy的想法直
接或间接地影响了早期的社交网络理论的研究,他被认为是六度分隔(six
degrees of separation)理论的最早提出者\cite{barabasi2003linked}。


\begin{table}
  \centering
  \begin{tabular}{cccp{38mm}}
    \toprule
    \textbf{文档域类型} & \textbf{Java类型} & \textbf{宽度(字节)} & \textbf{说明} \\
    \midrule
    BOOLEAN  & boolean &  1  & \\
    CHAR     & char    &  2  & UTF-16字符 \\
    BYTE     & byte    &  1  & 有符号8位整数 \\
    SHORT    & short   &  2  & 有符号16位整数 \\
    INT      & int     &  4  & 有符号32位整数 \\
    LONG     & long    &  8  & 有符号64位整数 \\
    STRING   & String  &  字符串长度  & 以UTF-8编码存储 \\
    DATE     & java.util.Date & 8 & 距离GMT时间1970年1月1日0点0分0秒的毫秒数 \\
    BYTE\_ARRAY & byte$[]$ & 数组长度 & 用于存储二进制值 \\
    BIG\_INTEGER & java.math.BigInteger & 和具体值有关 & 任意精度的长整数 \\
    BIG\_DECIMAL & java.math.BigDecimal & 和具体值有关 & 任意精度的十进制实数 \\
    \bottomrule
  \end{tabular}
  \caption{测试表格}\label{table:test2}
\end{table}

1961年,Michael Gurevich在社会学家Ithiel de Sola Pool的指导下完成了他的
博士论文,对社交网络进行了实验性的研究。随后,数学家Manfred Kochen与
Sola Pool一道在他们的手稿《Contacts and Influences》中对这些实验结果做
了分析,发现在美国人口中,任意两个人之间通常只需不超过两个中间人即可互
相联系\cite{pool1978}。1973年他们又利用计算机,基于Gurevich的数据,用
Monte Carlo法做了模拟,并证实了该结论\cite{pool1978},从而为心理学家
Stanley Milgram后来的发现打下了基础。

\section{网络结构的重要指标}

在刻画复杂网络结构的统计特性上有三个重要的指标:平均路径长度(average
  path length)、聚类系数(clustering coefficient)和度分布(degree
  distribution)。事实上,Watts和Strogatz提出小世界网络模型的初衷,就是
想建立一个既具有类似随机图的较小的平均路径长度,又具有类似规则网络的较
大的聚类系数的网络模型。

\subsection{平均路径长度}

\begin{definition}[节点之间的距离]
网络中两个节点$i$和$j$之间的距离(distance)$d_{ij}$定义为连接这两个节点
的最短路径上的边数。
\end{definition}

\begin{definition}[直径]
网络中任意两个节点之间的距离的最大值称为该网络的直径,记为$D$,即
\begin{equation}\label{eq:dimension}
    D = \max_{i,j} d_{ij}
\end{equation}
\end{definition}


\begin{figure}[htbp]
  \centering
  \includegraphics[width= 0.5\textwidth]{jnuname.eps}\\
  \caption{测试插图}\label{fig:test3}
\end{figure}

\begin{definition}[平均路径长度]
网络的平均路径长度$L$定义为任意两个节点之间的距离的平均值,即
\begin{equation}\label{eq:avarage_path_lentgh}
    L = \frac{2}{N(N+1)}\sum_{i\geq j}d_{ij}
\end{equation}
其中$N$为网络节点数。网络的平均路径长度也称为网络的特征路径长度。
\end{definition}

注意,为了便于数学处理,在公式\eqref{eq:avarage_path_lentgh}中包含了节
点到其自身的距离(该距离为零)。如果不考虑节点到其自身的距离,那么公式
\eqref{eq:avarage_path_lentgh}的右端需要乘以因子$(N+1)/(N-1)$。在实际应
用中,该差别可以忽略不计。

\subsection{聚类系数}

\begin{table}
  \centering
  \begin{tabular}{cccp{38mm}}
    \toprule
    \textbf{文档域类型} & \textbf{Java类型} & \textbf{宽度(字节)} & \textbf{说明} \\
    \midrule
    BOOLEAN  & boolean &  1  & \\
    CHAR     & char    &  2  & UTF-16字符 \\
    BYTE     & byte    &  1  & 有符号8位整数 \\
    SHORT    & short   &  2  & 有符号16位整数 \\
    INT      & int     &  4  & 有符号32位整数 \\
    LONG     & long    &  8  & 有符号64位整数 \\
    STRING   & String  &  字符串长度  & 以UTF-8编码存储 \\
    DATE     & java.util.Date & 8 & 距离GMT时间1970年1月1日0点0分0秒的毫秒数 \\
    BYTE\_ARRAY & byte$[]$ & 数组长度 & 用于存储二进制值 \\
    BIG\_INTEGER & java.math.BigInteger & 和具体值有关 & 任意精度的长整数 \\
    BIG\_DECIMAL & java.math.BigDecimal & 和具体值有关 & 任意精度的十进制实数 \\
    \bottomrule
  \end{tabular}
  \caption{测试表格}\label{table:test3}
\end{table}

在图论中,聚类系数(clustering coefficient)是用来描述一个图中的顶点之间
结集成团的程度的系数。具体来说,是一个点的邻接点之间相互连接的程度。许
多大规模的实际网络都具有明显的聚类效应。例如生活社交网络中,你的朋友同
时也是朋友的概率会随着网络规模的增加而趋向于某个非零常数。这意味着这些
实际的复杂网络并不是完全随机的,而是在某种程度上具有类似于社会关系网络
中“物以类聚,人以群分”的特性。

\begin{figure}[htbp]
  \centering
  \includegraphics[width= 0.5\textwidth]{jnuname.eps}\\
  \caption{测试插图}\label{fig:test4}
\end{figure}

集聚系数分为整体与局部两种。整体集聚系数可以给出一个图中整体的集聚程度
的评估,而局部集聚系数则可以测量图中每一个结点附近的集聚程度。

\begin{definition}[整体聚类系数]
整体集聚系数的定义建立在闭三点组(邻近三点组)之上。假设网络中有一部分
节点是两两相连的,那么可以找出很多个“三角形”,其对应的三点两两相连,
称为闭三点组。除此以外还有开三点组,也就是之间连有两条边的三点(缺一条
  边的三角形)。这两种三点组构成了所有的连通三点组。整体集聚系数定义为
一个网络中所有闭三点组的数量与所有连通三点组(无论开还是闭)的总量之比,
即
\[
    C_{total}=\frac{3\times G_{\triangle}}{3 \times G_{\triangle} + G_{\wedge}}
\]
其中$C_{total}$表示网络的整体聚类系数,$G_{\triangle}$表示该网络中闭三
点组的个数,$G_{\wedge}$表示该网络中开三点组的个数\cite{luce1949method}。
\end{definition}

对图中具体的某一个点,它的局部集聚系数$C_i$表示与它相连的点抱成团(完全
  子图)的程度。Watts与Strogatz在1998年的论文
\cite{watts1998smallworld}中首次引入了这个概念,用以判别一个图是否是小
世界网络。

\begin{table}
  \centering
  \begin{tabular}{cccp{38mm}}
    \toprule
    \textbf{文档域类型} & \textbf{Java类型} & \textbf{宽度(字节)} & \textbf{说明} \\
    \midrule
    BOOLEAN  & boolean &  1  & \\
    CHAR     & char    &  2  & UTF-16字符 \\
    BYTE     & byte    &  1  & 有符号8位整数 \\
    SHORT    & short   &  2  & 有符号16位整数 \\
    INT      & int     &  4  & 有符号32位整数 \\
    LONG     & long    &  8  & 有符号64位整数 \\
    STRING   & String  &  字符串长度  & 以UTF-8编码存储 \\
    DATE     & java.util.Date & 8 & 距离GMT时间1970年1月1日0点0分0秒的毫秒数 \\
    BYTE\_ARRAY & byte$[]$ & 数组长度 & 用于存储二进制值 \\
    BIG\_INTEGER & java.math.BigInteger & 和具体值有关 & 任意精度的长整数 \\
    BIG\_DECIMAL & java.math.BigDecimal & 和具体值有关 & 任意精度的十进制实数 \\
    \bottomrule
  \end{tabular}
  \caption{测试表格}\label{table:test4}
\end{table}

\begin{definition}[局部聚类系数]


\begin{figure}[htbp]
  \centering
  \includegraphics[width= 0.5\textwidth]{jnuname.eps}\\
  \caption{测试插图}\label{fig:test5}
\end{figure}

假设网络中的一个节点$i$有$k_i$条边与其他节点相连,这$k_i$个节点称为节点
$i$的邻居。显然,在这$k_i$个节点之间最多可能有$k_i(k_i-1)/2$条边。而这
$k_i$个节点之间实际存在的边数$E_i$和总的可能的边数$k_i(k_i-1)/2$之比就
定义为节点$i$的聚类系数(clustering coefficient)$C_i$,即
\begin{equation}\label{eq:clustering_coefficient}
    C_i = \frac{2E_i}{k_i(k_i-1)}
\end{equation}
从几何特性上看,上式的一个等价定义为:
\begin{equation}\label{eq:clustering_coefficient_triangle}
    C_i = \frac{\text{与节点$i$相连的三角形的数量}}{\text{与节点$i$相连
        的三元组的数量}}
\end{equation}
其中,与节点$i$相连的三元组是指由节点$i$和其两个邻居节点构成的组合。
\end{definition}

\begin{figure}[htbp]
  \centering
  \includegraphics[width= 0.5\textwidth]{jnuname.eps}\\
  \caption{测试插图}\label{fig:test6}
\end{figure}

知道了一个图里的每一个顶点的局部集聚系数后,可以计算整个图的平均集聚系
数。这个概念也是Watts与Strogatz在1998年的论文
\cite{watts1998smallworld}中引入的:

\begin{definition}[平均聚类系数]
平均聚类系数定义为所有顶点的局部集聚系数的算术平均数,即
\begin{equation}
    \bar{C} = \frac{1}{n}\sum_{i=1}^{n} C_i.
\end{equation}
\end{definition}

\subsection{度分布}

\begin{definition}[度]
无向网络中,节点$i$的度(degree)$k_i$定义为与该节点连接的其他节点的数目。
有向网络中,节点的度分为出度(out-degree)和入度(in-degree)。节点的出度是
指从该节点指向其他节点的边的数目,节点的入度是指从其他节点指向该节点的
边的数目。
\end{definition}

直观上看,一个节点的度越大就意味着这个节点在某种意义上越“重要”。

\begin{definition}[平均度]
网络中所有节点$i$的度$k_i$的平均值称为网络的平均度,记为$Avg{k}$。
\end{definition}

\begin{definition}[度分布]
网络中节点的度的分布状况可用分布函数$P(k)$来描述:
\begin{equation}\label{eq:degree_distribution}
    P(k) = \text{一个随机选定的节点的度恰好是$k$的概率}
\end{equation}
$P(k)$称为该网络的度分布。
\end{definition}

\section{小世界网络}

\begin{definition}[小世界网络]
若网络的平均路径长度和网络的节点数目的对数成正比,即
\[
  L_{G} \propto \log N
\]
其中$N$是节点数目,则称这样的网络为``小世界网络''。
\end{definition}
\blindtext

\section{基于小世界理论的数据中心网络}
\blindtext
\subsection{DS小世界模型}
\Blindtext
\subsection{DS小世界模型的平均网络距离}
\Blindtext
\subsection{高维小世界网络的负载能力与容错分析}
\Blindtext
\section{SIDN网络模型}
\Blindtext
\subsection{SIDN模型构建}
\Blindtext
\subsection{路由策略}
\Blindtext
\section{逻辑节点内部资源分配}
\Blindtext
\subsection{问题描述}
\Blindtext
\subsection{全局路由算法}
\Blindtext
\section{SIDN模型参数分析}
\Blindtext
\subsection{连通性分析}
\Blindtext
\subsection{平均路由跳数}
\Blindtext
\subsection{总网络带宽}
\Blindtext
\subsection{容错能力}
\Blindtext
\section{仿真结果}
\subsection{平均路由跳数}
\Blindtext
\subsection{总网络带宽}
\Blindtext
\subsection{容错能力}
\Blindtext
\section{小结}
\Blindtext

%%%%%%%%%%%%%%%%%%%%%%%%%%%%%%%%%%%%%%%%%%%%%%%%%%%%%%%%%%%%%%%%%%%%%%%%%%%%%%%
\chapter{随机网络模型}\label{chapter_random}
\section{随机网络背景与研究现状}
\Blindtext
\section{WarpNet网络模型构建}\label{sec:warpnet_construction}
\subsection{网络结构}
\Blindtext
\subsection{双层拓扑结构}
\Blindtext
\section{路由算法}
\subsection{基于flood的路由发现算法}
\Blindtext
\section{网络性能分析}
\blindtext
\subsection{连通性与互联通率}
\Blindtext
\subsection{路由跳数}
\Blindtext
\subsection{总带宽}
\Blindtext
\subsection{故障对网络的影响}
\Blindtext
\section{仿真分析}
\Blindtext
\section{实验验证}
\Blindtext
\section{小结}
\blindtext

%%%%%%%%%%%%%%%%%%%%%%%%%%%%%%%%%%%%%%%%%%%%%%%%%%%%%%%%%%%%%%%%%%%%%%%%%%%%%%%
\chapter{无尺度网络模型}\label{chapter_scalefree}
\section{无尺度网络背景与研究现状}
\Blindtext
\section{基于数据中心的无尺度网络运用}
\subsection{无尺度网络在数据中心运用的分析}
\Blindtext
\subsection{无尺度网络与最大度}
\Blindtext
\subsection{带最大度约束的无尺度构造算法}
\Blindtext
\section{模型分析}
\subsection{度-度相关性分析}
\Blindtext
\subsection{聚类性分析}
\Blindtext
\section{基于数据中心的无尺度网络模型构建分析}
\blindtext
\subsection{节点结构与网络性能的关系}
\Blindtext
\section{平衡因子的参数调整}
\blindtext
\subsection{$q=0$的情况}
\Blindtext
\subsection{$q>0$的情况}
\Blindtext
\subsection{$q<0$的情况}
\Blindtext
\section{TPSF模型构建与性能分析}
\blindtext
\subsection{平均路由距离}
\Blindtext
\subsection{总网络带宽}
\Blindtext
\subsection{容错能力}
\Blindtext
\section{仿真结果}
\subsection{平均路由跳数}
\Blindtext
\subsection{总网络带宽}
\Blindtext
\subsection{容错能力}
\Blindtext
\section{小结}
\blindtext

%%%%%%%%%%%%%%%%%%%%%%%%%%%%%%%%%%%%%%%%%%%%%%%%%%%%%%%%%%%%%%%%%%%%%%%%%%%%%%%
\chapter{网络模型验证框架}\label{chapter_experiments}
\section{引言}
\Blindtext
\section{模拟平台实现}
\Blindtext
\subsection{系统结构}
\Blindtext
\subsection{构造模块}
\Blindtext
\subsection{虚拟网卡实现}
\Blindtext
\subsection{控制核心}
\Blindtext
\subsection{受控模块}
\Blindtext
\subsection{分布式虚拟交换网络}
\Blindtext
\section{验证结果}
\Blindtext
\subsection{真实环境验证}
\Blindtext
\subsection{海量虚拟化的检测与控制}
\Blindtext
\section{小结}
\blindtext

%%%%%%%%%%%%%%%%%%%%%%%%%%%%%%%%%%%%%%%%%%%%%%%%%%%%%%%%%%%%%%%%%%%%%%%%%%%%%%%
% 学位论文的正文应以《结论》作为最后一章
\chapter{结论}\label{chapter_concludes}

本文在第\ref{chapter_smallworld}章中,通过考虑数据中心网络布局构建中的最大度限制
问题,提出了符合数据中心网络基本要求的DS小世界模型,并分析了它的性质。随后提出
SIDN,将DS模型映射到具体的网络结构中,并分析了所构成网络的平均直径、网络总带宽、
对故障的容错能力等各项网络性能。

分析与仿真实验证明,SIDN网络具有很好的扩展能力,网络总带宽与网络规模成
近似线性增长的关系;具有很强的容错能力,链路损坏与节点损坏几乎无法破坏
网络的联通性,故障率对网络性能的影响与破坏节点/链路占总资源比率线性相关。

随后在第\ref{chapter_scalefree}章中,分析了无尺度网络在数据中心网络构建应用中的
理论方面问题。对Scafida \cite{gyarmati2010scafida}文中所述在最大度限制的情况下运
用BA算法构造的网络并不会损失无尺度性质的观点,进行了深入的分析,并指出了该论点的
局限性。

在给出了在引入节点最大度限制之后,利用分治和递归的思想,对无尺度网络
进行多层构建,对所构造的网络进行度-度相关性,以及聚类性分析。

\begin{table}
  \centering
  \begin{tabular}{cccp{38mm}}
    \toprule
    \textbf{文档域类型} & \textbf{Java类型} & \textbf{宽度(字节)} & \textbf{说明} \\
    \midrule
    BOOLEAN  & boolean &  1  & \\
    CHAR     & char    &  2  & UTF-16字符 \\
    BYTE     & byte    &  1  & 有符号8位整数 \\
    SHORT    & short   &  2  & 有符号16位整数 \\
    INT      & int     &  4  & 有符号32位整数 \\
    LONG     & long    &  8  & 有符号64位整数 \\
    STRING   & String  &  字符串长度  & 以UTF-8编码存储 \\
    DATE     & java.util.Date & 8 & 距离GMT时间1970年1月1日0点0分0秒的毫秒数 \\
    BYTE\_ARRAY & byte$[]$ & 数组长度 & 用于存储二进制值 \\
    BIG\_INTEGER & java.math.BigInteger & 和具体值有关 & 任意精度的长整数 \\
    BIG\_DECIMAL & java.math.BigDecimal & 和具体值有关 & 任意精度的十进制实数 \\
    \bottomrule
  \end{tabular}
  \caption{测试表格}\label{table:test5}
\end{table}

表\ref{table:test5}用于测试表格。随后分析了无尺度网络构造过程中,交换机节点与数
据节点的角色区别,分析了两者在不同比率下形成的网络形态,以及对网络性能造成的影响。

通过理论分析和仿真实验,分析并找出比率因子q的最佳取值。此外,无尺度现象
的引入提高了网络的聚类系数,从而在不失灵活性可靠性的基础上,进一步提升
了网络的性能。

在第\ref{chapter_random}章中,将关注点转移到交换机本身。由于图论难以描述数据中心
网络中的交换设备,因此放弃基于图的抽象模型,转而基于多维簇划分的思想,提出并设计
了WarpNet网络模型。

该网络模型突破了基于图描述的局限性,并对网络的带宽等指标进行理论分析并
给出定量描述。最后对比了理论分析、仿真测试结果,并在实际物理环境中进系
真实部署,通过6节点的小规模实验以及1000节点虚拟机的大规模实验,表明该模
型的理论分析、仿真测试与实际实验吻合,并在网络性能、容错能力、伸缩性灵
活性方面得到了进一步的提升。

在第\ref{chapter_experiments}章中,针对网络模型研究这一类工作的共性,设计构造通
用验证平台系统。以海量虚拟机和虚拟分布式交换机的形式,实现了基于少量物理节点,对
大规模节点的模拟。其模拟运行的过程与真实运行在实现层面完全一致,运行的结果与真实
环境线性相关。除为本文所涉若干网络模型提供验证外,可进一步推广到更为广泛的领域,
为各种网络模型及路由算法的研究工作,提供分析、指导与验证。

% 参考文献。应放在\backmatter之前。
% 推荐使用BibTeX,若不使用BibTeX时注释掉下面一句。
\nocite{*}
\bibliography{bachelor}
% 不使用 BibTeX
%\begin{thebibliography}{2}
%
%\bibitem{deng:01a}
%{邓建松,彭冉冉,陈长松}.
%\newblock {\em \LaTeXe{}科技排版指南}.
%\newblock 科学出版社,书号:7-03-009239-2/TP.1516, 北京, 2001.
%
%\bibitem{wang:00a}
%王磊.
%\newblock {\em \LaTeXe{}插图指南}.
%\newblock 2000.
%\end{thebibliography}

%%%%%%%%%%%%%%%%%%%%%%%%%%%%%%%%%%%%%%%%%%%%%%%%%%%%%%%%%%%%%%%%%%%%%%%%%%%%%%%
% 致谢,应放在《结论》之后
\begin{acknowledgement}
  首先感谢我的母亲韦春花对我的支持。其次感谢我的导师陈近南对我的精心指导和热心帮助。接下来,
  感谢我的师兄茅十八和风际中,他们阅读了我的论文草稿并提出了很有价值的修改建议。

  最后,感谢我亲爱的老婆们:双儿、苏荃、阿珂、沐剑屏、曾柔、建宁公主、方怡,感谢
  你们在生活上对我无微不至的关怀和照顾。我爱你们!
\end{acknowledgement}

%%%%%%%%%%%%%%%%%%%%%%%%%%%%%%%%%%%%%%%%%%%%%%%%%%%%%%%%%%%%%%%%%%%%%%%%%%%%%%%
\end{document}
